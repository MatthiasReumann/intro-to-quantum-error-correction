\section{Magic State Distillation}\label{sec:magic-states}

Another approach to achieving universal quantum computing was proposed by Bravyi and Kitaev in 2005~\cite{bravyi_universal_2005}. Their idea is to shift the problem from the implementation of a transversal non-Clifford gate (for our current discussion the $T$ gate) to the purification, also \textit{distillation}, of \textit{magic states}.

\smallskip
\noindent
Equation~\ref{eq:magic-state} defines one particular magic state.
\begin{equation}\label{eq:magic-state}
    \ket{A^{\pi/4}} = \frac{1}{\sqrt{2}}(\ket0 + e^{i\frac{\pi}{4}}\ket1)
\end{equation}
We now illustrate how to apply the following gate by ``consuming'' the $\ket{A^{\pi/4}}$ state. 
\begin{equation}
    \begin{pmatrix}
        1 & 0 \\
        0 & e^{i\pi{}/4} \\
    \end{pmatrix}
\end{equation}
The careful reader might have noticed that this is the $T$ gate. Let $\ket{\psi} = \alpha\ket0 + \beta\ket1$ be the state on which we want to apply the $T$ gate. Then, the initial state $\ket{\psi} \otimes \ket{A^{\pi/4}}$ is 
\begin{equation}
    \alpha\ket{00} + \alpha{}e^{i\frac{\pi}{4}}\ket{01} + \beta\ket{10} + \beta{}e^{i\frac{\pi}{4}}\ket{11}
\end{equation}
Now, apply a $CNOT$ gate where the control is the first qubit, i.e. the $\ket{\psi}$ state. The resulting state is
\begin{equation}
    \alpha\ket{00} + \alpha{}e^{i\frac{\pi}{4}}\ket{01} +  \beta\ket{11} + \beta{}e^{i\frac{\pi}{4}}\ket{10}
\end{equation}
We can simplify this term by re-grouping the first and second qubits and multiplying the second term with a global phase.
\begin{equation}
    \begin{aligned}
        \bigl(\alpha\ket0 + \beta{}e^{i\frac{\pi}{4}}\ket1\bigr) \otimes \ket0\\
        + e^{i\frac{\pi}{4}}\bigl(\alpha\ket0 + \beta{}e^{-i\frac{\pi}{4}}\ket1\bigr) \otimes \ket1
    \end{aligned}
\end{equation}
Thus by measuring the second qubit, we end up with either the desired state $T\ket{\psi}$ or have to apply a corrective $\frac{\pi}{2}$ rotation, i.e. a $S$ gate. Figure~\ref{fig:magic-state-t} depicts the complete procedure as a circuit.
\begin{figure}[h]
    \[
        \Qcircuit @C=1.5em @R=1em {
            \lstick{\ket{\psi}} & \ctrl{1} & \gate{S}\cwx[1]  &\rstick{T\ket{\psi}} \qw\\ 
            \lstick{\ket{A^{\pi/4}}}    & \targ     & \meter    & \\ 
        }
    \]
    \caption{Circuit consuming a magic state $\ket{A^{\pi/4}}$ to apply a $T$ gate on $\ket{\psi}$.}\label{fig:magic-state-t}
\end{figure}

\smallskip 

\noindent
We are left to address the question of how to create such magic states. First, note that $\ket{A^{\pi/4}} = TH\ket0$. Hence, if we protect the state $\ket{\psi}$ by encoding it in the Steane code the problem of the non-transversal logical $T$ gate arises again. It is worth mentioning that the circuit in Figure~\ref{fig:magic-state-t} consists only of Clifford gates which we can perform fault-tolerantly thanks to the Steane code. As in the previous section, the transversal logical $T$ gate of the $[[15,1, 3]]$ {Reed-Muller} code is useful. Here, we will use it to distill a single logical $\ket{A^{\pi/4}}$ in the Steane code.
\begin{enumerate}
    \item Encode $15$ logical qubits of the Steane code in the {Reed-Muller} code as $\ket{+_L}$
    \item Apply the transversal $T$ gate of the {Reed-Muller} code
    \item Perform error detection (and correction)
    \item Decode the {Reed-Muller} code to yield a single $\ket{A^{\pi/4}_L}$ in the Steane code
\end{enumerate}
As {code-switching}, this approach requires $105$ physical qubits to distill one logical magic state for the application of one logical $T$ gate. An in-depth analysis of the {time-resource-cost} of magic state distillation is beyond the scope of this paper. Thus, we refer the reader to Ref.~\cite{litinski_magic_2019}.

Lastly, it is worth mentioning that recent work demonstrated the encoding of magic states using a four-qubit code on an actual quantum chip~\cite{gupta_encoding_2024}. This result shows that quantum hardware slowly but surely catches up with the theory.