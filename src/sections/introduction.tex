\section{Introduction}

Quantum computing is a hot topic. Feynman's dreams of a quantum computer in 1982~\cite{feynman1982simulating} are today's promises of companies such as IBM, Google, and Microsoft~\cite{ibm_ibm_2024,google_quantum_ai_our_2024,azure_quantum_2024}. The ultimate goal is to run algorithms on these hardware devices. Similar to World War 2, applications in cryptography demonstrate the usefulness of quantum computing technology~\cite{preskill_quantum_2023}. Namely, Shor's prime factoring algorithm and its consequences on current cryptographic systems~\cite{shor_factoring_1997}. Over time, a wide range of algorithms with speed-ups over their classical counterparts emerged. Another critical field of research lies between the advancements in quantum hardware and the invention of quantum algorithms: quantum error correction. Its task is to protect delicate qubits from the adverse effects of errors. Error sources include qubit\hyph{}environment interactions, so-called \textit{decoherence}, and faulty gates.

In this paper, we review the fundamental concepts of quantum error correction in an intuitive and easily digestible way. For the content we primarily rely on already existing, but math-heavy, introductions such as Ref.~\cite{gottesman_introduction_2009} and textbooks such as Ref.~\cite{nielsen00}. Nevertheless, we always reference source materials as well as contemporary research results for a deeper dive into the literature. A main achievement of this paper is the supplementation of existing resources with graphical intuition.