\section{Summary}\label{sec:conclusion}

In this paper, we began by introducing the challenges that quantum error correction faces.
We then showed how each challenge can be overcome and reviewed the theoretical foundations 
and key ideas of quantum error correction along the way. After that, we reviewed some of the 
fundamentals of {fault-tolerant} quantum computing. We especially emphasized on the {non-transversal} 
$T$ gate of the Steane code and possible solutions for that problem - namely {code-switching} and
magic state distillation.

\smallskip 
\noindent
This is by no means a complete introduction to quantum error correction and {fault-tolerant} quantum computation (and was never meant to be). Nevertheless, after finishing this paper the reader should have a basic and intuitive understanding of its concepts.

\smallskip 
\noindent
A possible route for the curious reader is to study the referenced source material, which should now be relatively easier, or continue with topics such as topological quantum computing. For an introduction of the latter see Ref.~\cite{fujii_quantum_2015}.