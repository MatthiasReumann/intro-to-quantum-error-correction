\section{Challenges}

Even though quantum error correction often borrows ideas from its classical counterpart, quantum error correction faces the following unique challenges.

\vspace{0.75em}
\noindent
\textbf{No-Cloning Theorem} The no-cloning theorem states that no unitary operation $U$ exists that copies an unknown quantum state $\ket{\psi}$. Consequently, quantum analogs of classical repetition codes, where one copies individual bits and takes a majority vote, are infeasible.
\begin{equation*}
    \Qcircuit @C=1.5em @R=1em {
        \lstick{\ket{\psi}}  & \multigate{1}{U}  & \rstick{\ket{\psi}} \qw \\ 
        \lstick{\ket0}       & \ghost{U}         & \rstick{\ket{\psi}} \qw }
\end{equation*}

\vspace{0.5em}
\noindent
\textbf{Continous Errors} A qubit can be in any superposition of the basis states $\ket{0}$ and $\ket{1}$.
\begin{equation}
    \alpha\ket{0} + \beta\ket{1}
\end{equation}
where $\alpha{}$ and $\beta{}$ are complex numbers that fulfill the constraint $|\alpha{}|^2 + |\beta{}|^2 = 1$. Thus, a quantum computer has a continuum of states. Ordinary classical computers are binary - a bit is either $0$ or $1$, even after a bit-flip error. In comparison, it seems like quantum error correction requires an infinite amount of precision to detect all possible errors.

Let's look at a concrete and practically relevant example. Imagine that a quantum computer implements the following rotation operation.
\begin{equation}
    \begin{pmatrix}
        1 & 0 \\
        0 & e^{i\theta}
    \end{pmatrix}
\end{equation}
However, due to some miscalibration, it ends up rotating not by $\theta$ but $\theta{}+\delta{}$. Figure~\ref{fig:continous-errors} depicts this scenario on the \textit{bloch sphere} for a $\frac{\pi}{2}$ rotation. Throughout the computation, we apply this faulty operation many times. How small $\delta$ may be, the accumulation of errors ultimately leads to inaccurate results. A reasonable idea therefore is to detect and correct this error. The difficulty lies in the fact that $\delta$ can be any real number, leaving us with the problem stated in the previous paragraph.

\begin{figure}[h]
    \centering
    \begin{blochsphere}[radius=2cm,opacity=0,rotation=-105]
        \coordinate (origin) at (0,0);

        \drawLongitudeCircle[]{-105};
        \drawLatitudeCircle[style={opacity=0.4, dashed, thin}]{0};
        
        \labelLatLon{ket0}{90}{0};
        \labelLatLon{ket1}{-90}{0};
        \labelLatLon{ketminus}{0}{180};
        \labelLatLon{ketplus}{0}{0};

        \labelLatLon{ketplusi}{0}{-90};
        \labelLatLon{ketminusi}{0}{-270};

        \labelLatLon{ketpluserror}{0}{-25};

        \draw[opacity=0.4, dashed, thin] (ket0) -- (ket1);
        \draw[opacity=0.4, dashed, thin] (ketplus) -- (ketminus);
        \draw[opacity=0.4, dashed, thin] (ketplusi) -- (ketminusi);

        \node[above, inner sep=.25em] at (ket0) {\footnotesize$\ket0$};
        \node[below, inner sep=.25em] at (ket1) {\footnotesize$\ket1$};
        
        { 
            \setDrawingPlane{0}{0}
            \pic["\footnotesize $\frac{\pi}{2}$", current plane, draw, fill=rblue2!50, fill opacity=.5, text opacity=1, angle radius=2.2em, angle eccentricity=1.3,-Straight Barb]{angle=ketminus--origin--ketplus};
            \pic["\footnotesize $\delta$", current plane, draw, fill=rred!50, fill opacity=.5, text opacity=1, angle radius=2.2em,angle eccentricity=1.75]{angle=ketplus--origin--ketpluserror};
            
        }
        
        \draw[color=black, -latex] (origin) -- (ketpluserror);
    \end{blochsphere}
    \caption{Over-rotation by $\delta$ for a desired $\frac{\pi}{2}$ rotation on the bloch sphere.}\label{fig:continous-errors}
\end{figure}

\vspace{0.5em}
\noindent
\textbf{Measurements destroy quantum information} Measurements in quantum mechanics destroy the superposition of a quantum state. Contrary to the classical approach, we can not recover the original superposition after observing the state.

\smallskip
\noindent
Given these three challenges, quantum error correction seems like a daunting, if not impossible, task. Fortunately, as we will illustrate in the succeeding sections, each challenge can be overcome.